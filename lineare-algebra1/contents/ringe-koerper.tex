\section*{Ringe und Körper}

\subsection*{Ring mit Eins $(R,+,\cdot)$}
\begin{itemize}
	\item $(R,+)$ ist abel. Gruppe mit $0$
	\item $a\cdot(b\cdot c) = (a\cdot b)\cdot c$
	\item $\exists 1\in R: 1\cdot a = a \cdot 1 = a$
	\item $a \cdot (b+c)=a\cdot b + a\cdot c$ \\
		und $(b+c)\cdot a = b\cdot a + c \cdot a$
\end{itemize}
In kommutativen Ringen: $a\cdot b=b \cdot a$

\subsubsection*{Einheit in $R$}
$\exists a^{-1}\in R: a\cdot a^{-1}=a^{-1}\cdot a = 1$ \\
$\implies a$ ist Einheit \\
$R^* = \{a\in R \mid a \text{ ist Einheit}\}$ Gruppe

\subsubsection*{Unterring $S\le R$}
$1,~ a+b,~ -a,~ a\cdot b \in S$

\subsection*{Körper $(K,+,\cdot)$}
Komm. Ring mit Eins, wobei $K^* = K\setminus\{0\}$ \\
($(K\setminus\{0\},\cdot)$ ist abel. Gruppe)

\subsubsection*{Nullteilerfreiheit von Körpern}
$a,b\in K \setminus\{0\} \implies a\cdot b \neq 0$

\subsection*{$\Z_n=\Z / n\Z$}
$\Z_n$ ist ein komm. Ring mit Eins. \\
Wenn $n$ Primzahl, dann sogar Körper.

\subsection*{Der Polynomring $K[t]$}
$\displaystyle K[t] = \left\{\sum_{k=0}^n a_k\cdot t^k \mid a_k \in K, n\in\N \right\}$ \\
ist komm. Ring mit Eins $t^0$.

\subsubsection*{Koeffizientenvergleich}
$f=g \iff a_k = b_k \forall k$

\subsubsection*{Grad und Leitkoeffizient}
$\deg(f)=n$, $\deg(0)=-\infty$ \\
($f$ konstant, wenn $\deg(f)\le0$) \\
$\lc(f)=a_n$, $\lc(0)=0$

$\lc(f)=1 \lor f=0 \implies f$ normiert

\subsubsection*{Gradformeln}
\begin{itemize}
	\item $\deg(f+g) \le \max\{\deg(f),\deg(g)\}$
	\item $\deg(f\cdot g) = \deg(f)+\deg(g)$
\end{itemize}

\subsubsection*{Ideal $I$ von $K[t]$}
$I\subseteq K[t]$, $f,g\in I$ und $h\in K[t]$. \\
$f+g \in I$ und $h \cdot f \in I$

\subsubsection*{Division mit Rest}
$f,g \in K[t]\setminus\{0\}$
\begin{itemize}
	\item $\exists q,r \in K[t]: f=q\cdot g + r$ \\
		und $\deg(r) < \deg(g)$
	\item $\lambda\in K$ Nullstelle von $f$\\
		$\implies\exists q\in K[t]: f=q\cdot(t-\lambda)$
	\item Nullstellen $\le \deg(f)$
\end{itemize}

\subsubsection*{Fundamentalsatz der Algebra}
$\C$ ist algebraisch abgeschlossen, d.h.
jedes nicht-konstante Polynom in $\C[t]$ zerfällt über $\C$ in Linearfaktoren.

\subsubsection*{Irreduzible Polynome}
$f\in K[t]\setminus K$ ist irreduzibel, wenn \\
$f=g\cdot h \implies \deg(g)=0 \lor \deg(h)=0$

\section*{Gruppen \& Homomorphismen}

\subsection*{Gruppe $(G,\times)$}
\begin{itemize}
	\item $(g \times h) \times k = g \times (h \times k)$
	\item $\exists e \in G: \forall g\in G: e\times g = g$
	\item $\forall g\in G: \exists g^{-1}\in G: g^{-1} \times g = e$
\end{itemize}
In abelschen Gruppen: $g\times h = h \times g$

\subsubsection*{Rechenregeln}
\begin{itemize}
	\item $(g^{-1})^{-1}=g$ und $(gh)^{-1}=h^{-1}g^{-1}$
	\item $g^n \cdot g^m = g^{n+m}$ und $(g^m)^n = g^{m\cdot n}$
	\item $ga=gb\Rightarrow a=b \Leftarrow ag=bg$
\end{itemize}

\subsubsection*{Untergruppe $U\le G$}
$\forall u,v\in U: u\times v \in U$ und $u^{-1}\in U$

\subsubsection*{Durchschnitt von Untergruppen}
$\forall i\in I: U_i \le G$ \\
$\displaystyle \bigcap_{i\in I}U_i \le G$

\subsection*{Erzeugnis}
Gruppe $G$ und $M \subseteq G$.

$\displaystyle \langle M \rangle := \bigcap_{M\subseteq U \le G}U$ \\
$= \{g_1^{\alpha_1}\times\cdots\times g_n^{\alpha_n} \mid g_i\in M,\alpha_i\in\Z\} $

\subsubsection*{Zyklische Gruppen}
Gruppe heißt zyklisch, wenn sie von einem Element erzeugt werden kann.

\subsection*{Gruppenhomomorphismus $\varphi:G \to H$}
$(G, \times)$ und $(H, *)$ Gruppen. \\
$\forall g,g'\in G: \varphi(g\times g')=\varphi(g) * \varphi(g')$

\subsubsection*{Komposition}
$\varphi_1: G_1 \to G_2,\varphi_2: G_2 \to G_3$ Grp.hom. \\
$\implies \varphi_2 \circ \varphi_1: G_1 \to G_3$ Grphom. 

\subsubsection*{Eigenschaften}
\begin{itemize}
	\item $\varphi(e_G)=e_H$ und $\varphi(g^{-1})=(\varphi(g))^{-1}$
	\item $\varphi$ bij. $\implies \varphi^{-1}$ Grp.hom.
	\item Untergruppen bleiben erhalten
	\item $\im(\varphi):=\varphi(G) \le H$
	\item $\Ker(\varphi) := \varphi^{-1}(e_H) \le G$
	\item $\varphi$ injektiv $\iff \Ker(\varphi)=\{e_G\}$
\end{itemize}

\subsection*{Die symmetrische Gruppe $\S_n$}
$\S_n := \Sym(\{1,\dots,n\})$\\
$=\{\sigma \mid \sigma \text{ ist Permutation}\}$\\
($\sigma : \{1,\dots,n\}\to\{1,\dots,n\}$ bijektiv)

\subsubsection*{Mächtigkeit der $\S_n$}
Die $\S_n$ enthält genau $n!$ Elemente.

\subsubsection*{Zyklen und Transpositionen}
$\sigma = (a_1 \dots a_k)$ ist \textit{k-Zyklus}:
Vertauscht $a_1$ bis $a_k$ zyklisch, lässt den Rest.

$\tau  = (i j)$ ist 2-Zyklus bzw. \textit{Transposition}.

\subsubsection*{Zyklenzerlegung}
Permutationen können als Produkt von
\begin{itemize}
	\item disjunkten Zyklen
	\item Transpositionen
	\item Transp. benachb. Zahlen
\end{itemize}
geschrieben werden.

\subsubsection*{Signum $\sgn$}
$\exists_1 \sgn : (\S_n, \circ) \to (\{1,-1\},\cdot) \in \Hom$ \\
$\sgn(\tau)=-1, \quad \sgn(\tau_1 \circ \dots \circ \tau_k)=(-1)^k$

\subsection*{Faktorgruppen}

\subsubsection*{Linksnebenklassen}
Gruppe $G$ und $U\le G$. $g,h\in G$.

$ g \sim h \iff g^{-1}h\in U,\quad \bar g = gU$\\
$ G/U = \{gU \mid g \in G \}, \quad |G:U| = |G/U| $

\subsubsection*{Satz von Lagrange}
Endl. Gruppe $G$ und $U \le G$.

$|G| = |U|\cdot|G:U|$

\subsubsection*{Ordnung eines Elements}
$G$ Gruppe, $g\in G$.

Ordnung $o(g):=|\langle g \rangle|$ teilt $|G|$.

Außerdem: $o(g)=\inf\{k>0 \mid g^k = e\}$

\subsubsection*{Faktorgruppe (G/U,\times)}
Abel. Gruppe $(G,\times)$ und $U \le G$.

$(G/U,\times)$ ist abel. Gruppe mit
Neutralem $\bar e$ und Inverse $\bar{g^{-1}}$.

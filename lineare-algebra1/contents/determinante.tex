\section*{Determinante}
$A\in\Mat_n(K)$

\subsection*{Leibniz-Formel}
$\displaystyle \det(A) := \sum_{\sigma\in\S_n} \sgn(\sigma)\cdot a_{1\sigma(1)}\cdot\dots\cdots a_{n\sigma(n)}$

\subsection*{Kleine Matrizen}
$2\times2$: $a\cdot d - b\cdot c$ \\
$3\times3$: Sarrus

\subsection*{Dreiecksmatrizen}
Für obere/untere Dreiecksmatrizen ist die Determinante
das Produkt der Diagonaleinträge.

\subsection*{Transponierte}
$\det(A) = \det(A^t)$

\subsection*{Determinante als Volumenform}

\subsubsection*{Multilineare Abbildungen}
$f:V^n\to W$ multilinear, wenn
$f$ in jedem Argument linear ist.

\subsubsection*{Alternierende multi. Abbildungen}
Wenn $f(x_1,\dots,x_n)=0$, sobald zwei Argumente gleich sind,
ist $f$ alternierend.

$f(x_{\sigma(1)},\dots,x_{\sigma(n)}) = \sgn(\sigma)\cdot f(x_1,\dots,x_n)$

\subsubsection*{Determinante als Volumenform}
$\det : \Mat_n(K)\to K$ ist alternierend multilinear
mit $\det(\mathbbm{1}_n)=1$.

Ist $f: \Mat_n(K)\to K$ alt. mult.,\\
dann $f(A)=f(\mathbbm{1}_n) \cdot \det(A)$

\subsection*{Determinante und Gauß}
Es gilt bei Änderung von $A$ auf $A'$:
\begin{itemize}
	\item Zeilen/Spalten tauschen:\\ $\det(A')=-\det(A)$
	\item Zeile/Spalte mit $\lambda$ skalieren:\\ $\det(A')=\lambda\det(A)$
	\item Zeilen/Sp. aufeinander addieren:\\ $\det(A')=\det(A)$
	\item Nullzeile/-spalte: $\det(A)=0$
	\item Gleiche Zeilen/Sp.: $\det(A)=0$
\end{itemize}

\subsection*{Determinantenmultiplikationssatz}
$\det(A \circ B)=\det(A) \cdot \det(B)$

\subsection*{Invertierbarkeit}
$A$ inv.bar $\iff \det(A) \neq 0$. \\
Dann $\det(A^{-1}) = (\det(A))^{-1}$

\subsection*{Kästchensatz}
Wenn $A=\mat{B & C \\ 0 & D}$
mit $B,D$ quadratisch.

$\det(A)=\det(B)\cdot\det(D)$

\subsection*{Adjunkte}
$A \in \Mat_n(K), n \ge 2$

\subsubsection*{Ersetzungs-/Streichungsmatrix}
$A_i(b)$ ersetzt die $i$-te Spalte von $A$ mit $b$.

$A_{ji}$ ist $A$ ohne $j$-te Zeile und $i$-te Spalte.

\subsubsection*{Kofaktor und Adjunkte}
$a^\#_{ij} := (-1)^{i+j}\cdot\det(A_{ji})$ \\
$A^\# := (a^\#_{ij})\in\Mat_n(K)$

\subsubsection*{Satz über die Adjunkte}
$A^\# \circ A = A \circ A^\# = \det(A) \cdot \mathbbm{1}_n$

\subsubsection*{Inverse über Adjunkte}
$A$ inv.bar $\displaystyle \implies A^{-1}=\frac{1}{\det(A)}\cdot A^\#$

\subsection*{Laplacescher Entwicklungssatz}
Entwickle $\det$ nach $i$-ter Zeile: \\
$\displaystyle \det(A)=\sum_{j=1}^n(-1)^{i+j}\cdot a_{ij} \cdot \det(A_{ij})$ \\
(Ersetze Laufindex durch $i$ für $j$-te Spalte)

Vorzeichen: Beachte Schachbrettmuster.

\subsection*{Cramersche Regel}
$A\in\Mat_n(K)$ inv.bar und $b\in K^n$.

$Ax=b$ hat eindtg. Lös. $x=(x_1,\dots,x_n)^t$.

$x_i = \frac{1}{\det(A)}\cdot\det(A_i(b))$

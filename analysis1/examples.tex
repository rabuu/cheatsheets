\documentclass[10pt,a4paper]{article}
\usepackage[ngerman]{babel}
\usepackage[T1]{fontenc}
\usepackage[dvipsnames]{xcolor}
%\usepackage[LY1,T1]{fontenc}
\usepackage{tikz}
\usetikzlibrary{shapes,positioning,arrows,fit,calc,graphs,graphs.standard}
\usepackage[nosf]{kpfonts}
\usepackage[top=2mm,bottom=3mm,left=2mm,right=2mm,headsep=2mm,includehead]{geometry}
\usepackage[framemethod=tikz]{mdframed}
\usepackage{multicol}
\usepackage{microtype}
\usepackage{pdfpages}
\usepackage{csquotes}
\usepackage{booktabs}
\usepackage{enumitem}

\usepackage{titlesec}
\newcommand{\sectionrule}{ \par \hrule \nopagebreak \bigskip }

\titleformat{\section}{\sectionrule\large\color{Mahogany}}{\thesection}{1em}{\scshape}
\titleformat{\subsection}{\color{RoyalPurple}}{\thesubsection}{1em}{}
\titleformat{\subsubsection}{}{\thesubsubsection}{1em}{\itshape}

\titlespacing*{\section}{0cm}{0.4\baselineskip}{0.4\baselineskip}
\titlespacing*{\subsection}{0cm}{0.3\baselineskip}{0.3\baselineskip}
\titlespacing*{\subsubsection}{0cm}{0.2\baselineskip}{0.2\baselineskip}

\newlength\tindent
\setlength{\tindent}{\parindent}
\setlength{\parindent}{0pt}
\renewcommand{\indent}{\hspace*{\tindent}}


\usepackage{amsmath}
\DeclareMathOperator{\Diff}{Diff}
\DeclareMathOperator{\supp}{supp}
\DeclareMathOperator{\OS}{OS}
\DeclareMathOperator{\US}{US}
\DeclareMathOperator{\ZS}{ZS}
\DeclareMathOperator{\OI}{OI}
\DeclareMathOperator{\UI}{UI}
\DeclareMathOperator{\arccot}{arccot}

\usepackage{blindtext}

\usepackage{hyperref}
\newcommand\nohyper[1]{\begin{NoHyper}#1\end{NoHyper}}

\usepackage{lastpage,fancyhdr}
\pagestyle{fancy}
\fancyhf{}
\fancyhead[L]{Rasmus Buurman}
\fancyhead[C]{\textsc{Analysis I}, Beispiele}
\fancyhead[R]{Seite \thepage{}/\nohyper{\pageref{LastPage}}}
\renewcommand\headrulewidth{0.5pt}

\let\bar\overline

\begin{document}

\begin{multicols*}{2}
\section*{Substitutionsregel}
Berechne: $\displaystyle \int_a^b x\cdot\exp(x^2)dx$ für $a,b\in\mathbb{R}$

Substituiere:\\
$ \displaystyle
z = x^2, \quad
\varphi(x) = x^2, \quad
\varphi'(x) = 2x, \quad
f(z) = \frac{1}{2}\exp(z)$ \\

Damit erhalten wir:
\begin{align*}
	\int_a^b x\cdot\exp(x^2)dx &= \int_a^b f(\varphi(x))\cdot\varphi'(x)dx = \int_{\varphi(a)}^{\varphi(b)}f(z)dz \\
							   &= \int_{a^2}^{b^2}\frac{1}{2}\exp(z)dz = \frac{1}{2}\cdot\big(\exp(b^2)-\exp(a^2)\big)
\end{align*}

\hrule
\vspace{1em}

Berechne: $\displaystyle \int^y\tan(x)dx$

Substituiere: \\
$ \displaystyle
z = \cos(x), \quad
\varphi(x) = \cos(x), \quad
\varphi'(x) = -\sin(x), \quad
f(z) = -\frac{1}{z}$ \\

Damit erhalten wir:
\begin{align*}
	\int^y\tan(x)dx &= \int^y-\frac{-\sin(x)}{\cos(x)}dx = \int^y f(\varphi(x))\cdot\varphi'(x) = \int^{\varphi(y)}f(z)dz \\
					&= \int^{\cos(y)}-\frac{1}{z}dz = -\ln\big(\cos(y)\big)
\end{align*}

\hrule
\vspace{1em}

Ansatz für $\displaystyle \int^y\sqrt{1-x^2}dx$

Substituiere:\\
$ \displaystyle
z = \sin(x), \quad
\varphi(x) = \sin(x), \quad
\varphi'(x) = \cos(x), \quad
b=\arcsin(y)$

\[\int^y\sqrt{1-z^2}dz=\int^{\varphi(b)}f(z)dz=\int^b f(\varphi(x))\cdot\varphi'(x)dx\]

\section*{Rekursive Folgen}
Sei $a_0 = \frac{3}{2}$ und $a_{n+1} = (a_n - 1)^2 + 1$ für $n\in\mathbb{N}$.

Wir wollen den Grenzwert bestimmen.\\

\begin{enumerate}[label=\alph*.]
	\item Zeige, dass $a_n\ge1$ für alle $n\in\mathbb{N}$.

		Da für jedes $n\in\mathbb{N}$ gilt $(a_n-1)^2\ge0$, ist offensichtlich auch stets $a_{n+1}=(a_n-1)^2+1 \ge 1$.
		Auch ist der Anfangswert $a_0\ge1$.

	\item Zeige, dass die Folge monoton ist, hier $a_{n+1}\le a_n$ für $n\in\mathbb{N}$.

		Für $n=0$ gilt $a_1=\frac{5}{4} \le \frac{3}{2}=a_0$.\\
		Gelte nun die Aussage $a_{n+1}\le a_n$ für ein festes $n\in\mathbb{N}$.
		Da sowohl $a_n\ge1$ als auch $a_{n+1}\ge1$:
		\begin{align*}
			(a_{n+1}-1)^2 &\le (a_n-1)^2 \\
			\implies a_{n+2}=(a_{n+1}-1)^2+1 &\le (a_n - 1)^2+1=a_{n+1}
		\end{align*}
		Die Aussage folgt mit vollständiger Induktion.

	\item Zeige, dass die Folge beschränkt ist.

		Da $1\le a_n \le a_0$ für $n\in\mathbb{N}$ ist die Folge beschränkt.

	\item Bestimme den Grenzwert

		Da die Folge monoton und beschränkt ist, konvergiert sie.\\
		\[a \longleftarrow a_{n+1}=(a_n-1)^2+1 \longrightarrow (a-1)^2+1=a^2-2a+2\]
		Die Gleichung $a=a^2-2a+2$ hat die Lösungen $1$ und $2$.
		Aber da $2\ge a_0$ ist der Grenzwert $1$.
\end{enumerate}

\vfill\null
\columnbreak

\section*{Konvergenz von Folgen}
Die Folge $a_n = \frac{7n^2+3}{4n^2+n+1}$ ist konvergent.
Es gilt nämlich:
\[a_n=\frac{7+\frac{3}{n^2}}{4+\frac{1}{n}+\frac{1}{n^2}}\to\frac{7+0}{4+0+0}=\frac{7}{4}\]

\section*{Konvergenz von Reihen}

\subsection*{Majorantenkriterium}
Für $k\ge2$ ist die Reihe $\displaystyle \sum_{n=1}^\infty\frac{1}{n^k}$ konvergent.

Wir betrachten zunächst den Fall $k=2$.
\[a_n := \frac{1}{(n+1)^2} \le \frac{1}{n\cdot(n+1)} =: b_n\]
Die Reihe $\sum_{n=1}^\infty b_n$ ist eine konvergente Majorante von $\sum_{n=1}^\infty a_n$.

Mithilfe einer Indexverschiebung sehen wir, dass folgende Reihe auch konvergiert:
\[\sum_{n=1}^\infty\frac{1}{n^2}=1+\sum_{n=1}^\infty\frac{1}{(n+1)^2}\]
Für $k>2$ ist wegen $0\le\frac{1}{n^k}\le\frac{1}{n^2}$ ebendiese Reihe eine Konvergente Majorante von $\sum_{n=1}^\infty\frac{1}{n^k}$.

\section*{Potenzreihen}
\subsection*{Konvergenzradius}

Betrachte $\displaystyle \sum_{n=1}^\infty\frac{t^n}{n}$.

\subsubsection*{Cauchy-Hadamard}
$\displaystyle \lim_{n\to\infty}\left|\frac{n}{n+1}\right|=\lim_{n\to\infty}\left(1-\frac{1}{n+1}\right)=1-0=1$\\
$\implies r=1$

\subsubsection*{Betrachtung der Grenzen}
Für $x=-1$ erhalten wir die alternierende Harmonische Reihe $\sum_{n=1}^\infty\frac{(-1)^n}{n}$, die konvergiert.

Für $x=1$ erhalten wir die Harm. Reihe $\sum_{n=1}^\infty\frac{1}{n}$, die divergiert.


\section*{Grenzwerte von Funktionen}

Betrachte $f: \mathbb{R}\to\mathbb{R}: x\mapsto x^2$ und $a=3$.

Für eine Folge $(a_n)_{\mathbb{N}}$ mit $a_n\to3$ gilt:
\[f(a_n)=a_n^2=3 \longrightarrow 3\cdot3=9 \implies \lim_{x\to3}x^2=9\]

\hrule\vspace{1em}

Betrachte $\displaystyle f: \mathbb{R}\setminus1\to\mathbb{R}:x\mapsto\frac{x^2-1}{x-1}$ und $a=1$.

Da es in $\mathbb{R}\setminus1$ offenbar eine Folge gibt, die gegen $1$ konvergiert,\\
ist $a$ ein Häufungspunkt von $\mathbb{R}\setminus1$.\\
Sei nun $(a_n)_{\mathbb{N}}$ eine Folge in $\mathbb{R}\setminus1$ mit $a_n\to1$. Es gilt:
\[f(a_n)=\frac{a_n^2-1}{a_n-1}=a_n+1\to2 \implies \lim_{x\to1}f(x)=2\]

\vfill\null
\columnbreak

\section*{Taylor}
\subsection*{Taylorentwicklung des $\ln$}
\[\ln'(x)=\frac{1}{x} \qquad \ln^{(n)}(x)=(-1)^{n-1}\cdot\frac{(n-1)!}{x^n}\]

Das $n$-te Taylorpolynom im Entwicklungspunkt $1$ ist also:
\[T_{\ln,1}^n(x)=\sum_{k=0}^n\frac{\ln^{(k)}(1)}{k!}\cdot(x-1)^k=\sum_{k=1}^n(-1)^{k-1}\cdot\frac{(x-1)^k}{k}\]
Der Betrag der Ableitungen auf $[1,2]$ ist streng monoton fallend, so dass insb. $|\ln^{(n+1)}(c)\le|\ln^{(n+1)}(1)|=n!$
für $c\in[1,2]$.

Also gibt es für $x\in[1,2]$ ein $c$ zwischen $1$ und $x\le2$ mit
\[|\ln(x)-T_{\ln,1}^n(x)|=\frac{|\ln^{(n+1)}(c)|}{(n+1)!}\cdot|(x-1)|^{n+1}\le\frac{1}{n+1}.\]

Also konvergiert die Funktionenfolge der Taylorpolynome auf $[1,2]$ gleichmäßig gegen $\ln$,
und ebenso konvergiert sie dort gleichmäßig gegen die Taylorreihe.

Es gilt also für $x\in[1,2]$: $\displaystyle \quad \ln(x)=\sum_{n=1}^\infty(-1)^{n-1}\cdot\frac{(x-1)^n}{n}$

Für $x=1$ erhalten wir einen Wert für die alternierende Harmonische Reihe:
\[\sum_{n=1}^\infty\frac{(-1)^n}{n}=-\ln(2)\]

\section*{Kurvendiskussion}

\subsection*{Extremstellen}

\subsubsection*{Einfache Vorgehenweise}
\[f(x)=x^3-3x^2-1 \quad f'(x)=3x^2-6x \quad f''(x)=6x-6\]
Um mögliche Extremstellen zu finden, müssen wir die Nullstellen der ersten Ableitung finden.
Das ist für $x=0$ und $x=2$ der Fall.

In diesen Punkten schauen wir uns die zweite Ableitung an.
\begin{align*}
	f''(0)&=-6<0 \implies \text{lokales Maximum} \\
	f''(2)&=6>0 ~~\implies \text{lokales Minimum}
\end{align*}

\subsubsection*{Allgemeinere Vorgehensweise}
\[f(x)=x^4 \quad f'(x)=4x^3 \quad f''(x)=12x^2 \quad f'''(x)=24x \quad f''''(x)=24\]
Bei $a=0$ ist ein globales Minimum, was wir nur durch die allgemeinere Vorgehensweise der Extremstellenbestimmung
zeigen können, da
\[f'(0)=0 \quad f''(0)=0 \quad f'''(0)=0,\]
aber $f''''(0)=24>0$.


\subsection*{Wertebereich}
\[f(x)=\frac{x-1}{x^2+3} \qquad f'(x)=\frac{(x^2+3)-2x\cdot(x-1)}{(x^2+3)^2}\]
Es gilt $\displaystyle \lim_{x\to\pm\infty}f(x)=\lim_{x\to\pm\infty}\frac{\frac{1}{x}-\frac{1}{x^2}}{1-\frac{3}{x^2}}=0$.

Extremstellen der Ableitung bei $x=3$ und $x=-1$ mit $f(3)=\frac{1}{6}$ und $f(-1)=-\frac{1}{2}$.
Die Funktion ist stetig und hat keine Polstellen.
Also werden alle Werte in $W=[-\frac{1}{2},\frac{1}{6}]$ angenommen.

\vfill\null
\columnbreak

\section*{Konvergenz von Funktionenfolgen}

Die Folge $f_n:[0,1]\to\mathbb{R}: x\mapsto x^n$ konvergiert auf $[0,1]$ punktweise gegen die Funktion
\[f:[0,1]\to\mathbb{R}:x\mapsto\lim_{n\to\infty}x^n=\begin{cases}0, & x<1,\\ 1, & x=1, \end{cases}\]
aber die Folge konvergiert auf $[0,1]$ nicht gleichmäßig gegen $f$.
Betrachte dafür $\varepsilon=\frac{1}{4}$ und $n_\varepsilon\in\mathbb{N}$.

Setze $n := \max\{n_\varepsilon,2\}\ge n_\varepsilon$ und $x=\frac{1}{\sqrt{2}}\in[0,1)$. Dann:
\[|f_n(x)-f(x)|=\frac{1}{2}>\frac{1}{4}=\varepsilon\]

Beachte auch, dass $f$ nicht stetig in $1$ ist, obwohl alle $f_n$ stetig waren.

\section*{Stetigkeit}
\[f: \mathbb{R}\to\mathbb{R}: x\mapsto\begin{cases}0, & x=0, \\ \sin\left(\frac{1}{x}\right), & x\neq0 \end{cases}\]

Die Funktion ist in allen Punkten $a\in\mathbb{R}\setminus\{0\}$ stetig, in $0$ aber nicht.

Betrachte dazu die Nullfolge $a_n=\frac{1}{2n\pi+\frac{\pi}{2}}$.

\[f(a_n)=\sin(2n\pi+\frac{\pi}{2})\longrightarrow1\neq0\]

\section*{Differenzierbarkeit}
\[f:\mathbb{R}\to\mathbb{R}:x\mapsto\begin{cases}x^2, & x\notin\mathbb{Q}, \\ 0, & x\in\mathbb{Q}\end{cases}\]
$f$ ist in $0$ differenzierbar, weil
\[\Diff_{f,0}(x)=\begin{cases}\frac{x^2-0}{x-0}=x, & x\notin\mathbb{Q}, \\ \frac{0-0}{x-0}=0, & x\in\mathbb{Q}, \end{cases}\]
weshalb $\Diff_{f,0}(x)\to0$.\\

Die Funktion ist aber in \textit{keinem} anderen Punkt differenzierbar.

\end{multicols*}

\end{document}

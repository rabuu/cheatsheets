\documentclass[10pt,a4paper]{article}
\usepackage[ngerman]{babel}
\usepackage[T1]{fontenc}
\usepackage[dvipsnames]{xcolor}
%\usepackage[LY1,T1]{fontenc}
\usepackage{tikz}
\usetikzlibrary{shapes,positioning,arrows,fit,calc,graphs,graphs.standard}
\usepackage[nosf]{kpfonts}
\usepackage[top=2mm,bottom=3mm,left=2mm,right=2mm,headsep=5mm,includehead]{geometry}
\usepackage[framemethod=tikz]{mdframed}
\usepackage{microtype}
\usepackage{pdfpages}
\usepackage{csquotes}
\usepackage{booktabs}
\usepackage{enumitem}

\usepackage{titlesec}
\titlespacing*{\section}{0cm}{0.4\baselineskip}{0.4\baselineskip}
\titlespacing*{\subsection}{0cm}{0.3\baselineskip}{0.3\baselineskip}
\titlespacing*{\subsubsection}{0cm}{0.2\baselineskip}{0.2\baselineskip}

\usepackage{amsmath}
\DeclareMathOperator{\Diff}{Diff}
\DeclareMathOperator{\supp}{supp}
\DeclareMathOperator{\OS}{OS}
\DeclareMathOperator{\US}{US}
\DeclareMathOperator{\ZS}{ZS}
\DeclareMathOperator{\OI}{OI}
\DeclareMathOperator{\UI}{UI}
\DeclareMathOperator{\arccot}{arccot}

\usepackage{blindtext}

\usepackage{hyperref}
\newcommand\nohyper[1]{\begin{NoHyper}#1\end{NoHyper}}

\usepackage{lastpage,fancyhdr}
\pagestyle{fancy}
\fancyhf{}
\fancyhead[L]{Rasmus Buurman}
\fancyhead[C]{\textsc{Analysis I}, Beispiele}
\fancyhead[R]{Seite \thepage{}/\nohyper{\pageref{LastPage}}}
\renewcommand\headrulewidth{0.5pt}

\let\bar\overline

\begin{document}

\begin{minipage}{0.5\textwidth}
\subsection*{Substitutionsregel}
Berechne: $\displaystyle \int_a^b x\cdot\exp(x^2)dx$ für $a,b\in\mathbb{R}$

Substituiere:\\
$ \displaystyle
z = x^2, \quad
\varphi(x) = x^2, \quad
\varphi'(x) = 2x, \quad
f(z) = \frac{1}{2}\exp(z)$ \\

Damit erhalten wir:
\begin{align*}
	\int_a^b x\cdot\exp(x^2)dx &= \int_a^b f(\varphi(x))\cdot\varphi'(x)dx = \int_{\varphi(a)}^{\varphi(b)}f(z)dz \\
							   &= \int_{a^2}^{b^2}\frac{1}{2}\exp(z)dz = \frac{1}{2}\cdot\big(\exp(b^2)-\exp(a^2)\big)
\end{align*}

\hrule
\vspace{1em}

Berechne: $\displaystyle \int^y\tan(x)dx$

Substituiere: \\
$ \displaystyle
z = \cos(x), \quad
\varphi(x) = \cos(x), \quad
\varphi'(x) = -\sin(x), \quad
f(z) = -\frac{1}{z}$ \\

Damit erhalten wir:
\begin{align*}
	\int^y\tan(x)dx &= \int^y-\frac{-\sin(x)}{\cos(x)}dx = \int^y f(\varphi(x))\cdot\varphi'(x) = \int^{\varphi(y)}f(z)dz \\
					&= \int^{\cos(y)}-\frac{1}{z}dz = -\ln\big(\cos(y)\big)
\end{align*}

\hrule
\vspace{1em}

Ansatz für $\displaystyle \int^y\sqrt{1-x^2}dx$

Substituiere:\\
$ \displaystyle
z = \sin(x), \quad
\varphi(x) = \sin(x), \quad
\varphi'(x) = \cos(x), \quad
b=\arcsin(y)$

\[\int^y\sqrt{1-z^2}dz=\int^{\varphi(b)}f(z)dz=\int^b f(\varphi(x))\cdot\varphi'(x)dx\]

\hrule
\vspace{1em}

\subsection*{Rekursive Folgen}
Sei $a_0 = \frac{3}{2}$ und $a_{n+1} = (a_n - 1)^2 + 1$ für $n\in\mathbb{N}$.

Wir wollen den Grenzwert bestimmen.\\

\begin{enumerate}[label=\alph*.]
	\item Zeige, dass $a_n\ge1$ für alle $n\in\mathbb{N}$.

		Da für jedes $n\in\mathbb{N}$ gilt $(a_n-1)^2\ge0$, ist offensichtlich auch stets $a_{n+1}=(a_n-1)^2+1 \ge 1$.
		Auch ist der Anfangswert $a_0\ge1$.

	\item Zeige, dass die Folge monoton ist, hier $a_{n+1}\le a_n$ für $n\in\mathbb{N}$.

		Für $n=0$ gilt $a_1=\frac{5}{4} \le \frac{3}{2}=a_0$.\\
		Gelte nun die Aussage $a_{n+1}\le a_n$ für ein festes $n\in\mathbb{N}$.
		Da sowohl $a_n\ge1$ als auch $a_{n+1}\ge1$:
		\begin{align*}
			(a_{n+1}-1)^2 &\le (a_n-1)^2 \\
			\implies a_{n+2}=(a_{n+1}-1)^2+1 &\le (a_n - 1)^2+1=a_{n+1}
		\end{align*}
		Die Aussage folgt mit vollständiger Induktion.

	\item Zeige, dass die Folge beschränkt ist.

		Da $1\le a_n \le a_0$ für $n\in\mathbb{N}$ ist die Folge beschränkt.

	\item Bestimme den Grenzwert

		Da die Folge monoton und beschränkt ist, konvergiert sie.\\
		\[a \longleftarrow a_{n+1}=(a_n-1)^2+1 \longrightarrow (a-1)^2+1=a^2-2a+2\]
		Die Gleichung $a=a^2-2a+2$ hat die Lösungen $1$ und $2$.
		Aber da $2\ge a_0$ ist der Grenzwert $1$.
\end{enumerate}

\end{minipage}

\end{document}
